\documentclass[dvipdfmx,11pt,notheorems]{beamer}
%%%% 和文用 %%%%%
\usepackage{bxdpx-beamer}
\usepackage{pxjahyper}
\usepackage{minijs}%和文用
\usepackage{wrapfig}
\usepackage{txfonts}
\usepackage{minted}
\usepackage{ascmac}
\usepackage{tikz}
\usepackage{bussproofs}
\setminted{
    breaklines,
    mathescape=true,
    fontsize=\footnotesize,
}
\renewcommand{\kanjifamilydefault}{\gtdefault}%和文用

%%%% スライドの見た目 %%%%%
\usetheme{Madrid}
\usefonttheme{professionalfonts}
\setbeamertemplate{frametitle}[default][center]
\setbeamertemplate{navigation symbols}{}
\setbeamercovered{transparent}%好みに応じてどうぞ)
\setbeamertemplate{footline}[page number]
\setbeamerfont{footline}{size=\normalsize,series=\bfseries}
\setbeamercolor{footline}{fg=black,bg=black}
%%%%

%%%% 定義環境 %%%%%
\usepackage{amsmath,amssymb}
\usepackage{amsthm}
\theoremstyle{definition}
\newtheorem{theorem}{定理}
\newtheorem{definition}{定義}
\newtheorem{proposition}{命題}
\newtheorem{lemma}{補題}
\newtheorem{corollary}{系}
\newtheorem{conjecture}{予想}
\newtheorem*{remark}{Remark}
\renewcommand{\proofname}{}
%%%%%%%%%

%%%%% フォント基本設定 %%%%%
\usepackage[T1]{fontenc}%8bit フォント
\usepackage{textcomp}%欧文フォントの追加
\usepackage[utf8]{inputenc}%文字コードをUTF-8
\usepackage{otf}%otfパッケージ
\usepackage{bm}%数式太字
%%%%%%%%%%

\newcommand{\code}[1]{\mintinline{text}{#1}}
\newcommand{\red}[1]{{\color{red} #1}}
\newcommand{\blue}[1]{{\color{blue} #1}}
\newcommand{\defines}{\ensuremath{\overset{\text{def}}{\,\Longleftrightarrow\,}}}

\title[]{Term Rewriting System入門}%[略タイトル]{タイトル}
\author[]{服部桃子}%[略名前]{名前}
% \institute[]{}%[略所属]{所属}
\date{\today}%日付
\begin{document}

\begin{frame}[plain]\frametitle{}
\titlepage %表紙
\end{frame}

\begin{frame}{Contents}
  Handbook of Practical Logic and Automated Reasoningの4.5〜4.7節
  \begin{wrapfigure}{r}[20pt]{5cm}
    \includegraphics[width=100pt]{plar.jpg}
  \end{wrapfigure}

  \tableofcontents %目次
\end{frame}

\section{前回の問題設定}
\begin{frame}{前回のあらすじ}
  \begin{itemize}
    \item Egison入門
    \item 導出原理のEgisonでの実装
    \item 群の公理から、$e\cdot x = x,\quad x^{-1} \cdot x = e$を導きたい
    \begin{exampleblock}{群の公理}
      \vspace{-10pt}
      \begin{align*}
        (x \cdot y) \cdot z &= x \cdot (y \cdot z) \\
        x \cdot x^{-1} &= e \\
        x \cdot e &= x
      \end{align*}
    \end{exampleblock}
  \end{itemize}
\end{frame}

\section{項書き換え系(TRS)}

\begin{frame}{項書き換え系(TRS)とは}
  \begin{itemize}
    \item $R$ : 書き換え規則$l = r$の集合
    \item 規則に従って与えられた項を書き換えていく
    \begin{itemize}
      \item 左から右に書き換える
    \end{itemize}
    \item 任意の「等しい」項が同じ標準形を持つと仮定すると、

    $s \overset{?}{=} t$を示したいならば、
    $s, t$を標準形になるまで書き換えて比較すればよい
  \end{itemize}
  \begin{exampleblock}{例: 自然数の足し算}
    $R = \{O + x = x, \quad S(x) + y = x + S(y)\}$
    \begin{align*}
      S(O) + S(S(O))
      &\rightarrow O + S(S(S(O))) \quad ()\\
      &\rightarrow S(S(S(O))) \quad \blue{\leftarrow \text{標準形}}
    \end{align*}
  \end{exampleblock}
\end{frame}

\begin{frame}{項書き換え系(TRS)とは}
  うまくいかない場合もある
  \begin{exampleblock}{停止性がない場合}
    $R = \{x + y = y + x\}$
    \vspace{-5pt}
    \[
    a + b \rightarrow b + a \rightarrow a + b \rightarrow b + a \rightarrow a + b \rightarrow \cdots
    \]
  \end{exampleblock}

  \begin{exampleblock}{合流性がない場合}
    $R = \{x \cdot (y + z) = x \cdot y + x \cdot z,\, (x + y) \cdot z = x \cdot z + y \cdot z\}$
    \begin{align*}
      (a + b) \cdot (c + d)
      &\rightarrow a \cdot (c + d) + b \cdot (c + d) \\
      &\rightarrow^* (a \cdot c + a \cdot d) + (b \cdot c + b \cdot d) \\
      (a + b) \cdot (c + d)
      &\rightarrow (a + b) \cdot c + (a + b) \cdot d \\
      &\rightarrow^* (a \cdot c + a \cdot c) + (a \cdot d + b \cdot d)
    \end{align*}
  \end{exampleblock}
\end{frame}

\subsection{停止性, 合流性(confluence)}
\begin{frame}{停止性・合流性}
  \begin{screen}
    $X$ : 集合 \hspace{10pt} $\rightarrow$ : $X$上の簡約関係(2項関係)
  \end{screen}

  \begin{block}{定義}
    \vspace{-10pt}
    \begin{align*}
      x\text{が\red{正規形}である} &\defines
      \lnot (\exists y \in X. \, x \rightarrow y) \\
      \text{簡約関係が\red{停止性を持つ}} &\defines \text{無限の簡約列が存在しない} \\
      x\text{と}y\text{が\red{合流可能}} (x \downarrow y) &\defines \exists z.\, x \rightarrow^* z \land y \rightarrow^* z
    \end{align*}
  \end{block}
\end{frame}

\begin{frame}{停止性・合流性}
  \begin{block}{定義}
    簡約関係$\rightarrow$が\red{合流性を持つ}
    \vspace{-5pt}
    \[
      \defines
      \forall x, y, z.\, (x \rightarrow y \land x \rightarrow z) \Rightarrow
      \exists w. \, y \rightarrow^* w \land z \rightarrow^* w
    \]
  \end{block}

  \begin{table}[h]
    \begin{tabular}{ccc}
      ダイアモンド性 & 合流性 & 弱合流性 \\
      \begin{tikzpicture}[>=stealth, every edge/.style={->, draw}]
        \node at (1, 2) (x) {$x$};
        \node at (0, 1) (y) {$y$};
        \node at (2, 1) (z) {$z$};
        \node at (1, 0) (w) {$w$};

        \draw[->] (x) -- (y);
        \draw[->] (x) -- (z);
        \draw[->] (y) -- (w);
        \draw[->] (z) -- (w);
      \end{tikzpicture}
      &
      \begin{tikzpicture}[>=stealth, every edge/.style={->, draw}]
        \node at (1, 2) (x) {$x$};
        \node at (0, 1) (y) {$y$};
        \node at (2, 1) (z) {$z$};
        \node at (1, 0) (w) {$w$};

        \draw[->] (x) -- (y);
        \draw[->] (x) -- (z);
        \draw[->>][blue] (y) -- (w);
        \draw[->>][blue] (z) -- (w);
      \end{tikzpicture}
      &
      \begin{tikzpicture}[>=stealth, every edge/.style={->, draw}]
        \node at (1, 2) (x) {$x$};
        \node at (0, 1) (y) {$y$};
        \node at (2, 1) (z) {$z$};
        \node at (1, 0) (w) {$w$};

        \draw[->>][blue] (x) -- (y);
        \draw[->>][blue] (x) -- (z);
        \draw[->>][blue] (y) -- (w);
        \draw[->>][blue] (z) -- (w);
        \end{tikzpicture} \\
      \end{tabular}
    \end{table}
  \end{frame}

  \section{Lexicographic Path Orders}

\begin{frame}{}
\end{frame}

\section{Knuth-Bendix Completion}

\begin{frame}{}
\end{frame}

% \setcounter{framenumber}{\value{finalframe}}
\end{document}
