\documentclass[dvipdfmx,11pt,notheorems]{beamer}
%%%% 和文用 %%%%%
\usepackage{bxdpx-beamer}
\usepackage{pxjahyper}
\usepackage{minijs}%和文用
\usepackage{wrapfig}
\usepackage{txfonts}
\usepackage{minted}
\usepackage{ascmac} % screen
\usepackage{bussproofs}
% \usepackage{tikz}
\setminted{
    breaklines,
    mathescape=true,
    fontsize=\footnotesize,
}
\renewcommand{\kanjifamilydefault}{\gtdefault}%和文用

%%%% スライドの見た目 %%%%%
\usetheme{Madrid}
\usefonttheme{professionalfonts}
\setbeamertemplate{frametitle}[default][left]
\setbeamertemplate{navigation symbols}{}
\setbeamercovered{transparent}%好みに応じてどうぞ)
\setbeamertemplate{footline}[page number]
\setbeamerfont{footline}{size=\normalsize,series=\bfseries}
\setbeamercolor{footline}{fg=black,bg=black}
%%%%

%%%% 定義環境 %%%%%
\usepackage{amsmath,amssymb}
\usepackage{amsthm}
\theoremstyle{definition}
\newtheorem{theorem}{定理}
\newtheorem{definition}{定義}
\newtheorem{proposition}{命題}
\newtheorem{lemma}{補題}
\newtheorem{corollary}{系}
\newtheorem{conjecture}{予想}
\newtheorem*{remark}{Remark}
\renewcommand{\proofname}{}
%%%%%%%%%

%%%%% フォント基本設定 %%%%%
\usepackage[T1]{fontenc}%8bit フォント
\usepackage{textcomp}%欧文フォントの追加
\usepackage[utf8]{inputenc}%文字コードをUTF-8
\usepackage{otf}%otfパッケージ
\usepackage{bm}%数式太字
%%%%%%%%%%

\newcommand{\code}[1]{\mintinline{text}{#1}}
\newcommand{\red}[1]{{\color{red} #1}}
\newcommand{\blue}[1]{{\color{blue} #1}}
\newcommand{\defines}{\ensuremath{\overset{\text{def}}{\,\Longleftrightarrow\,}}}
\newcommand{\underred}[2]{\underset{\color{red} #1}{\underline{\color{red} #2}}}


\title[]{paramodulationとsuperposition}%[略タイトル]{タイトル}
\author[]{服部桃子}%[略名前]{名前}
\institute[]{}%[略所属]{所属}
\date{\today}%日付
\begin{document}

\begin{frame}[plain]\frametitle{}
\titlepage %表紙
\end{frame}

\begin{frame}\frametitle{Contents}
\tableofcontents %目次
\end{frame}

\section{前回までのあらすじ}
\begin{frame}{前回までのあらすじ}
  \begin{itemize}
    \item 第1回: 導出原理とEgisonでの実装
    \item 第2回: 項書き換え系の話
    \begin{itemize}
      \item 停止性、合流性
      \item Lexicographic Path Order
      \item Knuth-Bendix完備化
    \end{itemize}
  \end{itemize}
\end{frame}

\begin{frame}[fragile]{前回までのあらすじ / 導出原理}
  \begin{block}{定義: resolvent}
    \begin{itemize}
      \item resolventの定義: 節$c, c_1, c_2$に対し以下を満たすリテラル$L_1, L_2$が存在するとき、$c$を$c_1, c_2$のresolventという
      \begin{itemize}
        \item $L_1 \in c_1$かつ$L_2^* \in c_2$
        \item $L_1, L_2$の\red{最汎単一化子$\theta$}が存在し、$c = (c_1\theta \setminus \{L_1 \theta\}) \cup (c_2\theta \setminus \{L_2^* \theta\})$
      \end{itemize}
    \end{itemize}
  \end{block}

  \begin{exampleblock}{}
    \begin{prooftree}
      \AxiomC{$\{\red{P(a, b)}\}$}
      \AxiomC{$\{\red{\lnot P(a, x)}, Q(x)\}$}
      \RightLabel{{\footnotesize $\theta = [b / x]$}}
      \BinaryInfC{$\{Q(b)\}$}
    \end{prooftree}
  \end{exampleblock}

  \begin{block}{}
    \begin{minted}{racket}
(define $resolution ; resolventを求める
  (match-all-lambda (multiset (multiset literal))
    {[<cons <cons <lit ,1 $t> $xs>
       ; $t$との最汎単一化子が$\sigma$であるようなリテラル$\lnot s$
       <cons <cons <lit ,-1 (& <unify ,t $sigma> $s)> $ys>
         _>>
      {@(subst-clause sigma xs) @(subst-clause sigma ys)}]}))
    \end{minted}
  \end{block}
\end{frame}

\begin{frame}[fragile]{前回までのあらすじ / 群の性質を示す}
  \begin{itemize}
    \item 群の公理から群の性質を示す
    \begin{itemize}
      \item 等式は単なる2項述語なので等式の公理も必要
    \end{itemize}
    \begin{minted}{racket}
(define $p3
  {@axioms-of-groups
   @axioms-of-equality
   @congruence-groups
   ; (negation of) the goal: e * x = x
   {<Lit -1 <Compound "=" <Compound "*" {e e1}> e1>>}})

(solve p3) ; => 現実的な時間では終わらない
    \end{minted}
  \end{itemize}
  \begin{block}{目標}
    Egisonで実装した導出原理のプログラムを改良し、探索空間を減らす
    \begin{itemize}
      \item paramodulation
      \item superposition
    \end{itemize}
  \end{block}
\end{frame}

\section{paramodulation}
\begin{frame}{paramodulation}

  {\Large
  \begin{prooftree}
    \AxiomC{$C \lor s \doteq t$}
    \AxiomC{$D \lor P[s']$}
    \BinaryInfC{$\sigma (C \lor D \lor P[t])$}
  \end{prooftree}
  }
  \begin{itemize}
    \item 導出原理で等式を直接扱う手法
    \item 上のような規則を追加
    \begin{itemize}
      \item $s \doteq t$は「$s = t$または$t = s$」
      \item $\sigma$は$s$と$s'$の最汎単一化子 \, ($\sigma = \text{mgu}(s, s')$)
    \end{itemize}

    \item 等式の公理は\blue{reflexivity(反射性)以外は}不要になる
    \begin{itemize}
      \item functional reflexivityも必要とされる
      \begin{itemize}
        \item $f(x_1, \ldots, x_n) = f(x_1, \ldots, x_n)$
      \end{itemize}
    \end{itemize}
  \end{itemize}
\end{frame}

\subsection{理論}

\begin{frame}{paramodulation / completeness}
\begin{block}{定理: Refutational completeness}
$S$がnormal model\footnote{"="が等号として解釈されるモデル}を持たないならば、
$S$にreflexivityとfunctional reflexivityを加えたものからparamodulationと導出原理によって空節を導くことができる
\end{block}

\underline{証明(概略)}:

\vspace{5pt}
普通の導出原理の完全性を利用して、
「導出原理 + 等式の公理で導出可能」ならば「導出原理 + paramodulation + reflexivityで導出可能」を示す
\end{frame}

\begin{frame}{paramodulation / completeness}
  \begin{itemize}
    \item 例えば導出原理+(1-ary)functional congruenceを用いている場合...
    \begin{prooftree}
      \AxiomC{$\lnot (x = x') \lor f(\ldots, x, \ldots) = f(\ldots, x', \ldots)$}
      \AxiomC{$C \lor s = t$}
      \BinaryInfC{$C \lor f(\ldots, s, \ldots) = f(\ldots, t, \ldots)$}
    \end{prooftree}
    paramodulation + functional reflexivityで次のように示せる
    \begin{prooftree}
      \AxiomC{$f(\ldots, x, \ldots) = f(\ldots, \red{[x]}, \ldots) \lor C$}
      \AxiomC{$s = t$}
      \RightLabel{\blue{$\sigma = [s / x]$}}
      \BinaryInfC{$f(\ldots, \blue{s}, \ldots) = f(\ldots, \red{t}, \ldots) \lor C$}
    \end{prooftree}

    \vspace{10pt}
    \item 他、(1-ary) predicate congruenceを用いている場合や、対称性・推移性を用いている場合も同様の議論
  \end{itemize}
\end{frame}

\begin{frame}{paramodulation / 実装}

\end{frame}

\subsection{実装}
\section{superposition}
\subsection{理論}

\subsection{実装}

\begin{frame}{(余談)導出原理の最適化}

\end{frame}

\end{document}
